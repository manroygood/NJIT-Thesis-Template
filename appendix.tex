\oneappendix{Cell Signaling and Stuff}
\label{APP:A}

NJIT has an interesting rule about appendices. If you have more than one appendix, they should be numbered like A, B, and C. These are declared with the \verb#\chapter# command. See the files \texttt{appendixA.tex} and \texttt{appendixB.tex}.

However, if you have only one appendix, it should not be numbered. Removing the number manually leads to all sorts of problems in numbering. equations/figures/tables.  In order that numbering works properly, we have created the \verb#\oneappendix# command that will handle all this for you. Note that's what is used above.

Here's an interesting requirement about NJIT disserations! There has to be text at the beginning of the appendix before any figures or tables. Who knew?

\begin{align}
x^2+1 & = y.\\
x-1 & = y^2.
\end{align}

\begin{figure}[htbp]
\centering
\includegraphics[width=.5in]{msgcube}
\caption{Here's the MSG logo again.}
\label{fig:2} 
\end{figure}


\begin{figure}[htbp]
\centering
\includegraphics[width=.5in]{msgcube}
\caption{Once more with the MSG logo.}
\label{fig:3} 
\end{figure}

\vspace{0.5in}
\renewcommand{\baselinestretch}{0.8} %Change some spacing
{\ssp
\begin{table}[htbp]

\caption{Useful internet sites for cell signaling and protein structure.}
\label{table:WEBSITES}

\begin{center}
	\begin{tabular}
	{|p{0.2in}|
	p{2.5in}|
	p{2.7in}|} \hline 
		\rule{0pt}{1pt} & & \\
			~ & 
			\PBS\raggedright{{\em Web Site} (Cited March 25, 2001)} & 
			\PBS\raggedright{\em Description}  \vspace{1mm} \\ \hline
		\rule{0pt}{1pt} & & \\ 
			$1.$ & 
			\PBS\raggedright{http://vlib.org/Science/ Cell$\_$Biology/ signal$\_$transduction.shtml} &
			\PBS\raggedright{The WWW Virtual Library: Cell Biology---with 
				information on other signal transduction sites of interest} \vspace{1mm}\\
			$2.$ &
			\PBS\raggedright{http://www.grt.kyushu--u.ac.jp/spad/}  &
			\PBS\raggedright{Signaling Pathway Database--- contains diagrams of cell signaling pathways}  \vspace{1mm} \\
			$3.$ &
			\PBS\raggedright{http://geo.nihs.go.jp/csndb/} &  
			\PBS\raggedright{Cell Signaling Networks Database---a signal transduction database \cite{CSND:1999}} \vspace{1mm} \\
			$4.$ &
			\PBS\raggedright{http://www.sdsc.edu/kinases/} &
			\PBS\raggedright{The Protein Kinase Resource---data available on the enzymology, 
				genetics, molecular and structural properties of protein kinases \cite{PKR:1997}} \vspace{1mm}\\
			$5.$ &
			\PBS\raggedright{http://www.expasy.ch/sprot/}  & 
			\PBS\raggedright{SWISS PROT Database---contains protein 
				sequences with functional and structural information \cite{BAIROCH:2000}} \vspace{1mm} \\
			$6.$ &
			\PBS\raggedright{http://www.expasy.ch/prosite/} & 
			\PBS\raggedright{Prosite Pattern Database---contains 
				information on protein families and protein domain structure \cite{HOFMANN:1999}} \vspace{1mm}\\
			$7.$ &
			\PBS\raggedright{http://www.cbs.dtu.dk/databases/ PhosphoBase} &
			\PBS\raggedright{PhosphoBase---a database of phosphorylation sites in proteins \cite{KREEGIPUU:1999}}\vspace{1mm} \\
			$8.$ &
			\PBS\raggedright{http://www--lmmb.ncifcrf.gov/ phosphoDB}    & 
			\PBS\raggedright{Phosphoprotein Database---site dedicated to protein phosphorylation} \vspace{1mm} \\
			$9.$ &
			\PBS\raggedright{http://www.rcsb.org/pdb} & 
			\PBS\raggedright{PBD Brookhaven Crystallographic Database---a protein data 
				bank containing 3--d structural X-ray crystallographic data \cite{BERMAN:2000}} \vspace{1mm} \\
			$10.$ &
			\PBS\raggedright{http://www--nbrf.georgetown.edu/} & 
			\PBS\raggedright{Protein Information Resource--- maintains a protein sequence database, the PIR-International
				Protein Sequence Database \cite{BARKER:2001}} \vspace{1mm}\\
			$11.$ &
			\PBS\raggedright{http://www.ncbi.nlm.nih.gov/} & 
			\PBS\raggedright{National Center for Biotechnology Information} \vspace{1mm} \\ 
			$12.$ &
			\PBS\raggedright{http://www.ncgr.org/software/ pathdb} & 
			\PBS\raggedright{PATHDB: Metabolic Pathways Database---contains information on
			pathways relating to metabolism in plants.} \vspace{1mm} \\ \hline 
				
	\end{tabular}
\end{center}
\end{table}
\endsinglespace}
\renewcommand{\baselinestretch}{2} % Change the spacing back
